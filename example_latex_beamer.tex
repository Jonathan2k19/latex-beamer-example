\documentclass[t]{beamer}

%%%%%%%%%%%%%%%%%%%%%%%%%%%%%%%%%%%%%%%%%%%%%%%%%%%%%%%%%%%%%%%%%%%%%%%%%%%%%%%%
% Packages
%%%%%%%%%%%%%%%%%%%%%%%%%%%%%%%%%%%%%%%%%%%%%%%%%%%%%%%%%%%%%%%%%%%%%%%%%%%%%%%%
\usepackage[american]{babel}
\usepackage{%
    microtype,
    listings, lstautogobble, % code snippets
    xcolor, % syntax highlighting
    upquote, % straight quotes
    %hyperref, cleveref,
    %caption, subcaption,
    %amsmath, amssymb, mathtools,
    %graphicx, wrapfig,
    xurl % better line breaks in long URLs
}
\graphicspath{ {./figures/} }

\definecolor{codegreen}{rgb}{0,0.6,0}
\definecolor{light-bg}{rgb}{0.98,0.98,0.95}
\lstdefinestyle{mystyle}{
    backgroundcolor=\color{light-bg},
    commentstyle=\color{gray},
    keywordstyle=\color{blue},
    numberstyle=\tiny\color{gray},
    stringstyle=\color{codegreen},
    basicstyle=\ttfamily\footnotesize,
    breakatwhitespace=false,
    breaklines=true,
    captionpos=b,
    keepspaces=true,
    %numbers=left, numbersep=5pt,
    showspaces=false,
    showstringspaces=false,
    showtabs=false,
    tabsize=4,
    columns=flexible,
    % left-align code i.e. remove outer-most indentation:
    autogobble=true,
    % Uncomment to frame the listing:
    frame=single,
    % Uncomment to (roughly) align line numbers with left text border:
    %numbersep=1em,
    %xleftmargin=1em
}
\lstset{style=mystyle}

%%%%%%%%%%%%%%%%%%%%%%%%%%%%%%%%%%%%%%%%%%%%%%%%%%%%%%%%%%%%%%%%%%%%%%%%%%%%%%%%
% Theme
%%%%%%%%%%%%%%%%%%%%%%%%%%%%%%%%%%%%%%%%%%%%%%%%%%%%%%%%%%%%%%%%%%%%%%%%%%%%%%%%
\usetheme{metropolis}
\usecolortheme{dove}
\setbeamerfont{frametitle}{size=\Large}
\setbeamertemplate{navigation symbols}{}

% Footline
\setbeamercolor{footline}{bg=white, fg=gray}
\setbeamertemplate{footline}[frame number]

% Itemize
\setbeamertemplate{itemize item}[circle]
\setbeamertemplate{itemize subitem}[circle]
\setbeamertemplate{itemize subsubitem}[circle]

% TOC
\setbeamertemplate{section in toc}[sections numbered]
\setbeamertemplate{subsection in toc}{%
  \leavevmode\hspace{2em}%
  \textcolor{black}{$\bullet$}\hspace{0.5em}%
  \inserttocsubsection\par%
}

% Font
\usepackage[T1]{fontenc}
\usepackage[utf8]{inputenc}
\usefonttheme{professionalfonts}
\usepackage{lmodern}

% Highlight current (sub-)section in outline at start of new (sub-)section
\AtBeginSection[]{
    \begin{frame}
        \frametitle{Outline}
        \tableofcontents[currentsection, subsectionstyle=shaded]
    \end{frame}
}
\AtBeginSubsection[]{
    \begin{frame}
        \frametitle{Outline}
        \tableofcontents[currentsection, currentsubsection]
    \end{frame}
}

%%%%%%%%%%%%%%%%%%%%%%%%%%%%%%%%%%%%%%%%%%%%%%%%%%%%%%%%%%%%%%%%%%%%%%%%%%%%%%%%
% Title page
%%%%%%%%%%%%%%%%%%%%%%%%%%%%%%%%%%%%%%%%%%%%%%%%%%%%%%%%%%%%%%%%%%%%%%%%%%%%%%%%
\setbeamertemplate{title page}{%
  \begin{centering}
    \vspace{2cm}
    {\inserttitle\par}
    \vspace{0.5cm}
    {\insertsubtitle\par}
    \vspace{2cm}
    {\insertauthor\par}
    \vspace{0.5cm}
    {\insertdate\par}
  \end{centering}
}

\title{\Huge{%
    The Presentation
}}
\subtitle{\Large{%
    The Subtitle
}}
\author{\small{%
    The Author | The Institution or Occasion
}}
\date{\small{%
    \today
}}

\begin{document}
% Define colors for highlighting and fading
\definecolor{highlight}{rgb}{0.2, 0.6, 0.8}
\definecolor{faded}{gray}{0.6}

% Customize the item colors in the table of contents
\newcommand{\highlightitem}[1]{
  \only<\currentsection>{\textcolor{highlight}{#1}}
  \only<\notcurrentsection>{\textcolor{faded}{#1}}
}
% Title and Outline
\begin{frame}
    \titlepage
\end{frame}

\begin{frame}
    \frametitle{Outline}
    \tableofcontents % [pausesections]
\end{frame}

%%%%%%%%%%%%%%%%%%%%%%%%%%%%%%%%%%%%%%%%%%%%%%%%%%%%%%%%%%%%%%%%%%%%%%%%%%%%%%%%
% TODO: Your content comes here ...
%%%%%%%%%%%%%%%%%%%%%%%%%%%%%%%%%%%%%%%%%%%%%%%%%%%%%%%%%%%%%%%%%%%%%%%%%%%%%%%%
\section{Some Section}
\begin{frame}
    \frametitle{Introduction}
    This is some text with \textbf{bold} and \textit{italic} words\footnote{This is a footnote.}.\\
    \pause
    Here comes a formula:
    \[
        \int_0^\infty foo
    \]
    \pause
    \begin{itemize}
        \item Here comes an item
            \begin{itemize}
                \item And a subitem. Some long URL:
                    \url{https://tex.stackexchange.com/questions/11168/change-bullet-style-formatting-in-beamer}
                    \begin{itemize}
                        \item And a subsubitem. Some hy-phe-nation -- okay.
                    \end{itemize}
            \end{itemize}
        \pause
        \item Here comes another item
            \begin{enumerate}
                \item With numbered
                    \pause
                \item Items
            \end{enumerate}
    \end{itemize}
\end{frame}

\section{Some Other Section}
\begin{frame}[fragile]
    \frametitle{Here Is Some Code}

    To make code listings work, mark the frame as \verb$fragile$.
    \begin{lstlisting}[language=python]
        def foo(a, b, c):
            # printing a, b, c
            print(f'a={a}, b={b}, c={c}')
    \end{lstlisting}
\end{frame}

\subsection{Some Subsection}
\begin{frame}[fragile]
    \frametitle{Example: C Code}

    \begin{lstlisting}[language=c]
        int main(int argc, char **argv) {
            for (int i = 1; i < argc; i++)
                printf ("argument %u: %s\n", i, argv[i]);
            return 0;
        }
    \end{lstlisting}
\end{frame}

\subsection{Some Other Subsection}
\begin{frame}[fragile]
    \frametitle{Example: C Code ... Again ...}

    \begin{lstlisting}[language=c]
        int main(int argc, char **argv) {
            for (int i = 1; i < argc; i++)
                printf ("argument %u: %s\n", i, argv[i]);
            return 0;
        }
    \end{lstlisting}
\end{frame}
\end{document}
